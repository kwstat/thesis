
% compnorm.tex

\documentclass{report}
\usepackage{isuthesis,natbib}
%\usepackage{natbib}
\begin{document}
\author{Kevin Wright}

\section{Interval Arithmetic}

\cite{Moo66} was one of the early developers of the techniques Interval
Analysis.   

While it is not possible to implement true interval arithmetic on computers
(any more than it is possible to implement true scalar arithmetic), it is
possible to produce an implementation which guarantees an enclosure of all 
arithmetic operations.

We shall implement our interval arithmetic calculations using the BIAS/PROFIL
(Basic Interval Arithmetic Subroutines / Programmer's Runtime Optimized Fast
Interval Library)
package \cite{Knu93a, Knu93b} in C++.   The BIAS software is 
modelled after the
Basic Linear Algebra Subroutines (BLAS) software.  Specifically, careful
thought and testing has gone into the devlopment of the software to achieve
optimal speed and accuracy in the computed results.

While BIAS/PROFIL contains C++ classes for complex numbers and operations, we
have found it necessary to add an implementation of a class for interval
complex arithmetic.

The extension from complex arithmetic to interval complex arithmetic is not
quite as intuitive as the extension from real arithmetic in interval real
arithmetic.  One possible extension is via the use of disks in the 
complex plane, i.e. by the pair $(z,r)$ where $z \in C$ is the
center of the disk and $r$ is its radius.
See \cite{Nickel80} for a brief discussion.

More common is the representation
of an interval complex number as a rectangle.  Such a rectangle can be
represented by a pair of complex numbers denoting the lower left and upper
right corners as in \cite{RL71}, 
or as a pair of interverals $(r,i)$ where $r$ and $i$ are in the
real and imaginary coordinate directions.  These two representations of a
complex interval rectangle are equivalent, differing only in the 
software data structures used to support the implementation.  
We have chosen to use the $(r,i)$ representation, which seems to be natural to
use with the BIAS/PROFIL package.

Given this representation, arithmetic operations on interval complex numbers
can be implemented quite easily.  However, how to achieve
 the best implementation is not transparent.  This results from the
unfortunate characteristic that for rectangular complex interval numbers
$z_1$ and $z_2$, 
$$
S = \left\{ z_1 / z_2 | z_1 \in z_1, z_2 \in z_2 \right\}
$$
is generally not a rectangular region.

\cite{RL71} presented an algorithm for obtaining a small 
rectangular enclosure of $S$.  The scalar complex arithmetic
algorithm in PROFIL, which calculates $r = d/c$ and then
$$
\frac{a+bi}{c+di} = \frac{(a+br)+(b-ar)i}{c+dr}
$$
could also be used in hopes of achieving a fast answer.  In the context where
we use interval complex arithmetic, the intervals are degenerate (or nearly
so, given rounding).  This may explain why we have found the usual definition
of complex division, implemented with interval arguments, to give the best
results.

\section{Complex Error Function}

The error function of a complex number $z$ is defined as 
$$
erf(z) = \frac{2}{\sqrt{\pi}}\int_0^z e^{-t^2} dt
$$
and the complementary error function is given as $erfc(z) = 1 - erf(z)$.
A widely-used related function is the plasma dispersion function
\begin{equation}
w(z) = e^{-z^2}erfc(-iz)  \label{plasma}
\end{equation}.

Test:  The Faddeeva function \ref{plasma} is OK.

Let $\overline{z}$ denote the complex conjugate of $z$ and $i$ denote the
square root of $-1$.  Using the relations
$$
w(-z) = 2 e^{-z^2}-w(z)
$$
$$
w(\overline{z}) = \overline{w(-z)}
$$
it is possible to obtain $w(z)$ for values of $z$ over the entire complex plan
using the values of $w(z)$ for $z$ in the first quadrant.

Numerous techniques for computation of the complex error function have been
proposed.  See \cite{Weideman} for several references.

To successfully enclose the result of a computation, it is necessary to use
techniques which offer strict bounds on truncation errors.  Fortunately, such
bounds exist for many complex continued fractions.  

Let 

$$
f(z) = \frac{\sqrt\pi e^{z^2}}{z}erfc(z) =
\frac{2 e^{z^2}}{z}\int_z^\infty e^{-t^2} dt
$$
.

From results in \cite{JonesThron}, it is possible to represent $f(z)$ for all
$z$ with $Re(z) > 0$ by the continued fraction
$$
f(z) = \frac{1}{z} \left( \frac{1}{z+} \frac{1/2}{z+} \frac{1}{z+} 
\frac{3/2}{z+} \frac{2}{z+} \frac{5/2}{z+} \cdots \right)
$$.

Let $f_n(z), n = 1, 2, 3, \ldots$ represent the convergents of the continued
fraction and let $f(z)$ be the limit of this sequence.  \cite{JonesThron} show
that the continued fraction satisfies certain necessary conditions so that

$$
|f(z) - f_n(z)| \leq \left\{
\begin{array}{ll}
|f_n(z) - f_{n-1}(z)|                      &,  |\arg z| \leq \pi/2 \\
\sec(|\arg(z)|-\pi/2)|f_n(z) - f_{n-1}(z)| &,  \pi/2 < |\arg z | < \pi \\
\end{array}
\right.
$$
.

Since we are considering values of $z$ in the first quadrant, once our
algorithm stops at the $n^{th}$ iteration, the box $f_n(z)$ can be 
expanded by the amount $r = |f_n(z) - f_{n-1}(z)|$ in both the positive and
negative directions of the real and imaginary axes to obtain a guaranteed
enclosure for $f(z)$.


\section{Implementation Details}

For complex numbers $a_1, a_2, \ldots, b_1, b_2, \ldots$ we use the notation 
$$
\frac{a_1}{b_1 +} \frac{a_2}{b_2 +} \frac{a_3}{b_3 +} \cdots
$$ 
as a shorthand for the continued fraction
$$
\frac{a_1}{\displaystyle b_1+
   \frac{a_2}{\displaystyle b_2 + 
      \frac{a_3}{b_3 + \cdots}}}
$$
When the continued fraction is of finite length $n$, its value is denoted by
$f_n$, the $n^{th}$ convergent.

One method of evaluating continued fractions in the forward direction is
through the use of the system of second-order linear difference equations
$$
\begin{array}{c}
A_{-1} = 1, A_0 = 0, B_{-1} = 0, B_0 = 1,                 \\
A_n = b_n A_{n-1} + a_n A_{n-2}, n= 1,2,3,\ldots          \\
B_n = b_n B_{n-1} + a_n B_{n-2}, n= 1,2,3,\ldots          
\end{array}
$$
From these, the convergents may be calculated as $f_n = A_n/B_n$.
Forward evaluation of continued fractions has the advantage that the stopping
value of $n$ need not be determined in advance.  The backward recurrence
algorithm (not defined here) requires determination of $n$ before calculations
proceed.  According to \cite{JonesThron}, the backwards recurrence algorithm
may be preferred for computational purposes since the rounding error in
calculating $f_n$ is either bounded or else grows very slowly.

Not as good as forward.  Contraction.  Rescaling.


\section{Future work}

\cite{PoppeWijers} descibe a method which improves upon that of
\cite{Gautschi70} to a suggested relative accuracy of 14 digits.

\bibliographystyle{isuapalike}
\bibliography{/home/kwright/Thesis/Refs/kwright}
\end{document}


