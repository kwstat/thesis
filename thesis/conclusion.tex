%\bibliographystyle{apalike}

\chapter{GENERAL CONCLUSIONS}

%% \setcounter{chapter}{1}
%% \setcounter{figure}{0}
%% \setcounter{section}{0}
%% \setcounter{subsection}{1}

This dissertation has explored various ways in which interval analysis can be
utilized in statistical computing.  Interval analysis uses intervals instead
of real numbers as the basic units of calculation.  Interval analysis provides
a means to evaluate the range of a function over a given domain.  Using this
capability, it is possible to obtain numerical answers which are guaranteed to
be correct to a certain level of accuracy and to contain the true result.
Interval analysis is also able to guarantee finding global optimizers within
an initial box.

There are two broad topics covered in this dissertation as related to interval
analysis.  Several sections of the dissertation are focused on obtaining
numerical values which are of a guaranteed accuracy.  The application to
statistics is in the computation of critical points and tail probabilities of
several statistical distributions. For a bivariate chi-square distribution
considered in one section, a series expansion is used together with a bound on
the truncation error for the series.  By evaluating the finite series with
intervals and evaluating the truncation error with intervals, an interval
enclosure of the true probability is obtained.  To find critical points of a
distribution, one approach requires solving for the root of an equation.  Due
to the complicated form of the equation involved, the use of derivatives (as
in the Newton-Raphson algorithm) is avoided in favor of a derivative-free
root-finding method.  An interval secant algorithm (with Illinois
modification) is developed and used for finding critical points of the
distribution.  A bivariate $F$ distribution is also considered using similar
techniques with the added complication that each term in the series involves
the calculation of an interval Gaussian quadrature rule for evaluating a
numerical integral.  The methods prove successful for computing guaranteed
enclosures of the probabilities to several decimal places (at a minimum).  In
some cases the guarantee extends to more than a dozen accurate digits.  The
guaranteed values allowed for the discovery of errors in earlier published
tables.


A third section of the dissertation considers the global optimization
capabilities of interval analysis.  The EM algorithm is widely used in
statistics for estimating parameters of a model when data is missing.  The
original EM paper by \cite{DLR} has been cited in more than 2000 papers since
its publication and continues to be an active area of research.  Much of the
research about the EM algorithm has explored ways to speed up the rate of
convergence to a stationary point.

While enjoying enormous popularity, the EM algorithm, like many optimization
methods, generally only converges to a stationary point (not necessarily the
global optimum).  Using special properties of the EM algorithm, an interval
enclosure of the gradient of the loglikelihood is derived.  The enclosure of
the gradient is evaluated over an interval region.  Regions where the
enclosure of the gradient does not contain zero are therefore known to contain
no stationary points.  By beginning with an initial region and repeatedly
subdividing the region into ever smaller pieces, regions which are known to
not contain a stationary point are eliminated from consideration.  Any
stationary points of the loglikelihood are located in the union of the
undiscarded regions.  If the initial region is large enough, all stationary
points of the loglikelihood will be found.  No other known method is capable
of achieving the same result.

Interval analysis has been demonstrated to be an effective tool in
statistical computing.  The unique opportunities and challenges associated
with interval analysis promise both opportunities for future research and
future rewards.

%\renewcommand{\bibname}{\centerline{Bibliography}}
%\addcontentsline{toc}{section}{Bibliography}
%\bibliography{thesis}

